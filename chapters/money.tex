\chapter{Money Matters}
\label{chapter:money-matters}

Much of the operation of the organisation hinges on how it manages its money.  This chapter provides details on the processes involved in managing Student Robotics' money.  The monetary management system consists of the following elements:

\begin{enumerate}
\item A budget, which describes how money is allocated.
\item A system for approving spends against the budget.
\item An accounting system, which keeps records of all financial activity.
\item Taxes.
\item Fundraising, which brings money in.
\end{enumerate}

The above topics are discussed in detail in this chapter.


\section{The Budget}
\label{sec:budget}

The budget describes how the organisation's money is allocated.  It is comprised of a tree of directories containing ``budget lines''.  Any money that is spent by the organisation must be described by a budget line.    Each budget line has the following associated properties:

\begin{description}
\item[Name] This is a short name for the budget line.  It must be all lower-case, and may not include any whitespace characters.
\item[Value]  The monetary value of the budget line in GBP.
\item[Description] A description of what the money associated with this line is for.  This should contain information on how the value of the line was derived.
\item[Closed] This boolean indicates whether the line has been closed or not.  A closed line may no longer have any money spent against it.
\item[Roles] A list of the roles within the organisation that are able to either invoke spends against this budget line, or permit specific groups of volunteers to claim expenses against this line.
\item[Claimable] This boolean indicates whether it is possible for registered volunteers to claim expenses against this budget line when permitted by those in the roles associated with the line.
\end{description}

Budget lines are referred to by their path within the tree, using forward-slashes to separate the directory and budget line names -- e.g. \texttt{/sr2017/clothing}.  All budget lines that are entirely related to the Student Robotics competition programme should be contained within a directory named after the appropriate Student Robotics year.

In order to avoid situations in which suppliers know exactly how much the organisation has to spend on their goods or services, the contents of the budget must not be shared to the general public, and is only accessible to those who are registered as volunteers.

The budget always describes money that the organisation has already received.  This means that if there is an entry in the budget that is not closed, then it is available for spending.  The actual sum available depends on the amount that has already been spent against that budget line.  The total amount spent against a budget line must not exceed the total value of the line.

\subsection{Budget Update Procedure}

The budget is managed in a git repository.  The canonical git repository for this is hosted at \url{https://bitbucket.org/srobo/budget}.  The master branch of this canonical git repository contains the current budget.  All changes to the contents of the master branch of this git repository must be approved through a resolution of the trustees.

\subsection{Budgeting for the Future}

It is important for there to be a clear understanding of the organisation's monetary requirements for the future.  Therefore, a second branch, called \texttt{future}, is maintained within the budget's git repository.  This branch must contain the current budgetary requirements for the next 3 SR years after the current one.  This information will be used by our fundraisers to ensure that the project can continue.

\subsection{Budget Estimation}

The value of a budget line is an estimate of the amount that will be spent on the goods or services that it is for.  There will be error in that estimate, and so it is essential that this is taken into account when setting the value of a budget line.  Once an estimate has been reached for a budget line, 20\% must be included in the value to allow for error. 


\section{Spending Money}

There are three routes through which money leaves the Student Robotics bank account:

\begin{enumerate}
\item Expense claims by volunteers.
\item Direct purchases from suppliers.
\item Loan repayments.
\end{enumerate}

\subsection{Expense Claims}

When a volunteer spends money on behalf of Student Robotics, they may be later reimbursed through an expense claim.  An expense claim may only be made by a claimant when all of the following conditions are met:

\begin{enumerate}
\item The budget line that the claim is being made against is marked as claimable.
\item Before spending the money, the claimant received written approval to do so from one of the people in the roles associated with the relevant budget line.  This written approval is known as a ``claim approval'', and must either be in an email or on paper.
\item The volunteer provides receipts for the amount that they are claiming to the treasurers by email.
\item The claim is less than or equal to the maximum amount described in the claim approval.
\item The claim is for the goods or services described within the claim approval.
\end{enumerate}

Claims should be filed by emailing the treasurer email address, supplying the receipts mentioned above, the name of the budget line to claim against, and information regarding how the claim should be repaid (UK bank account number and sort code preferred).

A claim approval is valid for a maximum of six months.  If the claim is not made before the approval expires, then the claim may not be made.  In this case, the claimant may request another claim approval, but it is not guaranteed that the budget line will still be open at that point.

\subsubsection{Approving Claims}

Those in a budget line's roles may send claim approvals if the line is marked as claimable.  These claim approvals must contain the following:
\begin{enumerate}
\item The date.
\item The claimant's name.
\item The maximum amount that they may claim.
\item The budget line that they are claiming against.
\item What the claimant is purchasing (e.g. ``Train fare for visiting St Edwards.``).
\end{enumerate}

When a claim approval is sent, a copy of the approval must also be sent to the treasurers (ideally by CC'ing them in the email in which it is sent).  The treasurers will subsequently update the accounting system to record the amount that has been allocated to the claim.

The total sum of claim approvals against one budget line must not exceed the value of that line.  Since it will take some time for the treasurer to update the accounting system to reflect this notification, those able to write claim approvals against the budget line should coordinate to ensure that the budget line's limit is not exceeded.

\subsection{Direct Purchases}
Most of Student Robotics expenses go through direct purchases, as these provide the fastest method of getting money to a supplier, and do not result in a volunteer being out-of-pocket for any period of time.

The treasurers are the only people able to make direct purchases, and so requests for direct purchases must be sent to them.  Such requests must include:
\begin{enumerate}
\item The budget line that the purchase is being made against.
\item The total sum of the purchase.
\item Sufficient information for the purchase to be made (i.e. the company the purchase is being made from, the amount, the product, part numbers, invoices, etc.).
\end{enumerate}

Only the roles listed on a budget line may make the request for a direct purchase.

It may be the case that Student Robotics has long-running accounts or service agreements with suppliers.  For example, Student Robotics may have an arrangement with an external consultant, such as a solicitor or accountant.  It is important to note that the point at which such suppliers are instructed to do work is the point at which an expense is being created.  This is still a direct purchase from a supplier, and so all of the above requirements still apply.  When one wishes to engage in such a transaction, one must get a reasonable estimate of the expense that it will incur, which one will then use to form a \textbf{maximum} expense that will go into the purchase request sent to the treasurer.

\subsection{Loan Repayments}

When approved by the trustees, Student Robotics may be leant money by third parties.  This will be repaid those those third parties when the trustees resolve that it is the right time to do so.

\section{Accounting}

\section{Fundraising}

\section{Tax and Annual Reports}
