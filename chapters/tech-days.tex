\chapter{Tech Days}

\begin{itemize}
\item What a tech day is:
  \begin{itemize}
  \item Opportunity for teams to work on their robot for a full working day.  Teams generally work on them for 1 or 2 hours a week normally.
  \item Teams come to us, rather than mentors to team.
  \item Teams meet other teams.
  \end{itemize}


\item What should happen at tech days:
  \begin{itemize}
  \item Tech days should generally run from 9 or 10 AM to 4 or 5 PM.
  \item Teams must be provided with:
    \begin{itemize}
    \item Internet access
    \item Toilet facilities
    \item Table space, chairs.
    \item Refreshments: Water at a minimum.  SR volunteers must not be in the position of providing hot drinks: must be done by third party.
    \end{itemize}
  \end{itemize}

\item Some volunteers (\textit{how many!?}) must be present to assist teams -- these volunteers must meet the same criteria that apply to mentors.

\item  Organising the tech day:
  \begin{itemize}
  \item Provide advance notice to teams of at least 8 weeks.  PR must be informed too.
  \item Provide template risk assessment for teachers.
  \item Perform risk assessment.
  \item Date selection:
    \begin{itemize}
    \item Must be on a weekend, or alternatively a weekday in a school holiday, or bank holiday.
    \item Ensure that enough volunteers can attend.
    \end{itemize}
  \item Teams must register to attend.  \textit{How?}
  \end{itemize}

\end{itemize}
