\section{Kit Services}

\begin{enumerate}

\item The philosophy behind the kit:

  \begin{enumerate}

    \item The vast majority of the competitors will never have engineered any software, nor anything mechanical or electronic.  Asking them to build a robot from scratch, without giving them any support in these areas would, in most cases, not be useful.  It would not be enjoyable for them, as they would likely end up believing that they could not achieve anything.

    \item SR provides a kit to teams.  The purpose of this kit is to make it feasible for completely inexperienced teams to build robots capable of competing in the competition.  This kit consists of three main parts: mechanics, electronics and software.

    \item The kit must be able to be easily constructed into a ``minimum viable robot'', which is capable of competing within the game.  The kit must contain all of the parts for this minimum viable robot, including the mechanics, electronics and software.

Not only does this minimum viable robot remove any barrier to entry, it allows teams to focus on the elements of their robot that they are most interested in (and thus enjoy the most).  For example, a team may have more of an interest in developing the electronics for their robot.
      
    \item The construction process for assembling the kit into this minimum viable robot must be straightforward.  If someone can assemble a piece of flatpack furniture, then they should be able to assemble all of the minimum viable robot (including its electronics and software).

    \item The kit should not be designed just to create a minimum viable robot, but should consist of composable modules that can be used in robots designed by teams.  

    \item Teams must not be required to use the kit to compete.  They should be able to build their robot out of whatever they like, provided that it is safe.

    \item The kit should be enjoyable for teams to use.

    \item The kit should be as close to the state of the art as possible.

    \item The kit must be rugged and robust to accidental damage.
    \item{Consist of commercial off the shelf (COTS) parts where appropriate.}
  \end{enumerate}
\item Development philosophy
  \begin{enumerate}
    \item The kit generally consists of electronic hardware (including firmware), mechanical hardware, systems software and user software.
    \item Teams will directly interact with the electronic hardware, the mechanical hardware and the user software therefore its design and development must be well defined and controlled to ensure consistency and ease of use.
    \item The firmware and systems software is not directly visible to the teams. This must be robust and well tested, but ease of use for the teams is not a priority.
  \end{enumerate}
\item Development Process
  \begin{enumerate}
    \item Money is available for new kit development
    \item A new kit item proposal is required when starting to develop a new bit of team-facing kit. This proposal must include:
      \begin{enumerate}
        \item The name of the lead developer.
        \item A description of the item.
        \item The functionality it provides to teams.
        \item Example user stories to allow later assessment of suitability to fulfil requirements.
        \item An estimate of the development time required.
        \item A development plan showing design review points.
        \item In the case of electronic or mechanical hardware proposals:
          \begin{enumerate}
            \item A review of COTS products providing similar functionality and reasons why these are not suitable.
            \item The estimated development cost.
            \item The estimated production cost per kit.
            \item The estimated annual maintenance cost per kit (including consumable parts).
          \end{enumerate}
        \item In the case of systems or user facing software proposals:
          \begin{enumerate}
            \item The scope of the software and how/where it will interface with existing software.
            \item An estimate of new dependencies required (new libraries being pulled in, etc.)
            \item The testing methodology to be used.
          \end{enumerate}
      \end{enumerate}
    \item Proposals for new kit development must be reviewed and accepted by the \role{kit-coord} before money is allocated for development. No new bits of hardware or software will be included in the kit without approval from the \role{kit-coord}.
    \item Once development is completed a final kit addition proposal is required. This proposal must include:
      \begin{enumerate}
        \item A review of the original new kit item proposal showing how the developed item fulfils the user stories and provides the functionality specified.
        \item Links to the source of the hardware/software developed including any new documentation.
        \item In the case of electronic or mechanical hardware:
          \begin{enumerate}
            \item A reviewed estimated production cost per kit. This will be more accurate than the initial estimate due to the design being complete.
            \item A reviewed annual maintenance cost per kit.
            \item A manufacturing plan including a final deadline. If the new bit of kit is not fully manufactured, tested and entered into the inventory before this date then it will not ship in the next competition year.
          \end{enumerate}
        \item In the case of systems or user facing software:
          \begin{enumerate}
            \item A list of new dependencies and how they will be managed.
            \item A review of the testing carried out.
            \item A plan for ongoing testing.
          \end{enumerate}
      \end{enumerate}
    \item Final designs must be approved by the \role{kit-coord} before it enters into the kit and therefore, in the case of hardware, manufacturing.
    \item As with all engineering the hardware parts of the kit must be optimised for:
      \begin{enumerate}
        \item Ease of manufacture
        \item Ease of initial test
        \item Ease of annual testing
        \item Ease of use
        \item Robustness
        \item Cost
      \end{enumerate}
    \item Various documentation is required for kit:
      \begin{enumerate}
        \item Manufacturing (only applicable to hardware).
        \item Maintenance (only applicable to hardware).
        \item End user (applicable to hardware and software).
      \end{enumerate}
  \end{enumerate}

\item Manufacture
  \begin{enumerate}
    \item Responsibility of \role{hw-prod-coord} to get newly developed bits of kit manufactured/purchased ready for inclusion in the kit.
    \item Also need to manufacture new whole kits when increasing number of teams.
    \item Manufacturing only really applicable to electronic/mechanical kit components.
    \item Newly manufactured or purchased kit must be entered into the inventory.
    \item Kit being manufactured or purchased for the next Competition Programme Year must be ready for packing at least 2 weeks before the Kit Packing Event (TODO: link to section).
  \end{enumerate}

\item Production
  \begin{enumerate}
  \item Support
    \begin{enumerate}
      \item \role{kit-coord} provides Kit support to teams and other users of the Kit.
      \item Team support is primarily provided via the Forum.
    \end{enumerate}
  \item Inventory
    \begin{enumerate}
      \item Physical items of the kit are tracked in the Inventory. Each item is known as an asset.
      \item The Inventory allows for the condition (working, broken, unknown), location and other information of an asset to be recorded.
      \item All historic information for an asset is recorded.
      \item Each asset in the Inventory is identified with a unique asset code.
      \item The asset code is marked on the physical item either with a printed sticker or written by hand. (The preference is for a printed sticker that has both human and machine readable markings).
      \item Anyone (\emph{TODO: not actually anyone, need to define who}) may update the Inventory as assets are moved or their state changes. However the \role{kit-logistics-coord} is responsible for ensuring that the Inventory is a good representation of reality.
      \item When assets are disposed of the \role{kit-logistics-coord} is responsible for removing them from the Inventory.
      \item The Inventory also tracks some non-kit assets. These can be broken down into the following categories:
        \begin{enumerate}
          \item Kit development - tools and hardware for developing and testing the kit.
          \item Competition hardware - items such as arena walls and display computers. These are \textbf{only} for use at the competition and at no other event.
          \item Game hardware - the same types of assets as competition hardware but available for use throughout the year such as at tech days.
          \item General event hardware - items for use at all types of SR event.
          \item Miscellaneous - the small number of assets not covered by the above categories.
        \end{enumerate}
      \item The \role{kit-logistics-coord} is also responsible for the storage and shipping of non-kit assets in the Inventory.
    \end{enumerate}
  \item Kit lifecycle
    \begin{enumerate}
      \item Kit Collation Event
        \begin{enumerate}
          \item An annual two day event that ocurrs in July.
          \item All of the kits from the preceding Competition Programme Year are split into their constituent parts.
          \item The state of each piece of kit is assessed. Where applicable testing is carried out to verify the state.
          \item The Inventory is updated as the process progresses.
          \item The event must take place in a geographically central location.
          \item The event must be announced to volunteers at least 4 weeks prior to its date.
          \item Volunteers must register to attend to allow for work processes to be organised. They do not have to attend for both days.
          \item The venue must allow for the kit to be left in the workspace overnight. The workspace must be lockable.
          \item The \role{hw-prod-coord} is responsible for running the event.
        \end{enumerate}
      \item Kit Packing Event
        \begin{enumerate}
          \item An annual two day event that occurs in August at least four weeks after the Kit Collation Event.
          \item All of the kits for the following Competition Programme Year are packed ready for Kickstart.
          \item The Inventory is updated as the process progresses.
          \item The venue, announcement and registration requirements are the same as the Kit Collation Event requirements.
          \item The \role{hw-prod-coord} is responsible for running the event.
        \end{enumerate}
      \item Kit repair
        \begin{enumerate}
          \item Kit found to be faulty at the Kit Collation Event or otherwise known to be faulty should be repaired where possible.
          \item Any repairs must be completed prior to the Kit Packing Event.
          \item If a particular asset is beyond repair then a replacement must be purchased prior to the Kit Packing Event. In the case of SR custom electronics hardware replacements must be manufactured in the next manufacturing run.
        \end{enumerate}
    \end{enumerate}
  \end{enumerate}

\item Licensing requirements
  \begin{enumerate}

  \item The following must be licensed under a GPL-compatible license, as qualified by the \href{http://www.gnu.org/licenses/license-list.en.html#GPLCompatibleLicenses}{Free Software Foundation}:
    \begin{enumerate}
    \item All software distributed as part of the SR kit.
    \item All software executed on all servers run by SR. (Note that this does not include software run on third-party servers -- e.g. github.com -- for obvious reasons.)
    \end{enumerate}

  \item The designs of all hardware that is manufactured by SR for the kit must be licensed under at least one of the following licenses:
    \begin{enumerate}
    \item The Creative Commons Attribution-NonCommercial-ShareAlike 4.0 International license
    \item The Creative Commons Attribution-ShareAlike 4.0 International license
    \end{enumerate}
  \end{enumerate}


\item Requirements for kit shipping
  \begin{enumerate}
    \item{Whole kits shipped in white RUBs}
    \item{Batteries in battery bags}
    \item{\pounds500 insurance on whole-kit shipping}
    \item{Battery courier instructions (approved list, labels, etc)}
    \item{Link Kit Return section here when written.}
  \end{enumerate}
\item Policy on kit replacement
  \begin{enumerate}
    \item{\role{kit-logistics-coord} responsible for shipping replacement parts after request/authorisation from \role{kit-coord}.}
    \item{\role{local-team-coord} responsible for reporting broken/damaged kit to \role{kit-coord} for replacement.}
    \item{Any piece of kit replaced free-of-charge when damaged accidentally.}
    \item{Any piece of kit, except for batteries, replaced free-of-charge when damaged due to misuse/negligence in the first instance. Teams charged for further damage due to misuse/negligence.}
    \item{In the case of batteries damaged due to misuse/negligence no replacements will be provided and teams must dispose of the battery in question.}
  \end{enumerate}

\item Classes of kit
  \begin{enumerate}
    \item There are multiple classes of kit that fulfil distinct requirements. For each class a specific role is responsible for the allocation of said kits. For all classes of kit the \role{kit-logistics-coord} is responsible for the storage, tracking and shipping of the kits. The classes are as follows:
      \begin{enumerate}
        \item Production kit. This is the kit used by teams. The \role{team-coord} is responsible for allocation.
        \item Spare kit. This is kit ready to be shipped to teams as replacements. The \role{team-coord} is responsible for allocation.
        \item Development kit. This is kit used by the development team.  The \role{kit-coord} is responsible for allocation.
        \item Support kit. This is kit used by the support team to help teams and internal users of the kit with issues. The \role{kit-coord} is responsible for allocation.
        \item PR kit. This is kit available for public relations/exhibition use. The \role{pr-coord} is responsible for allocation.
        \item Local kit. This is kit available to \role{local-team-coord}s for training and occasional PR/advertising use. It may also be used at Tech Days to provide hands-on support. The \role{team-coord} is responsible for allocation.
      \end{enumerate}
    \item There are enough production kits to cater for the number of teams taking part in the competition programme.
    \item There are 0.125 times the number of teams of spare kits available.
    \item There are \emph{TODO: N} development, \emph{TODO: N} support, \emph{TODO: N} PR and \emph{TODO: N} local kits available.
    \item It is the responsibility of the \role{hw-prod-coord} to ensure that enough kits exist. Specifically the \role{hw-prod-coord} needs to ensure that there are more kits manufactured in preparation for the next competition programme year.
  \end{enumerate}
\end{enumerate}
