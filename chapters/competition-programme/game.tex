\section{Game Design}

This section covers the philosophy behind the Student Robotics game, and the
procedures for designing and revising it.

\subsection{Game philosophy}

An ideal game would have:
\begin{itemize}
	\item Low barriers to entry
	\item Wide scope for technical achievement translating into points
	\item Enjoyable for those taking part as well as those spectating
\end{itemize}

for the purpose of giving the competitors something back for the effort they
put in as much as possible, while not setting the bar so low that there is no
incentive to develop technically challenging robot features.

To elaborate on these principles, for all Student Robotics games the following
invariants should hold:
\begin{enumerate}
	\item Contact between robots should not be part of the game. While the occasional collision is inevitable, we are not Robot Wars, and should discourage and penalise unsportsmanlike interference with other robots.
	\item Interaction: the competition is at it's best when robots are aware of each other and react to the others actions. While not encouraging contact, robots should not be able to play in isolation.
	\item Robots should move with a purpose. Part of the competition is developing robot software: the game should discourage random robot behaviour, requiring some level of intelligence.
	\item Reproducible in schools. Each team should be able to build a mock arena to test their robot without too much effort.
	\item Effectively implementable. The arena props and scoring process should be practical, requiring neither great expense or genius, although this should be considered in proportion to what additional expense or complexity will add to the game.
\end{enumerate}

Whilst encouraging the active exploration or development of the following
features by the competitiors, most commonly in the form of rewarding such
behaviours with additional points:
\begin{enumerate}
	\item Technical excellence: robots should be able to benefit from the development of non-trivial software, mechanical or electronic features.
	\item Variety: there should be multiple ways to solve the competition
		problem, avoiding all robots looking and behaving the same.
	\item Strategy: A robot that is able to better understand it's
		environment and plan ahead should score well.
	\item Robustness: the game and arena environment should not baby sit
		the competitors or otherwise make their lives easy. Dealing with
		uncertain and hazardous (to clumsy robots) environment is what
		makes the game interesting
\end{enumerate}

Finally, the game should be well grounded. The competitors are not all
programming geniuses or electronics wizards, and the aim of the competition is
to encourage them to study such things. Those who have only had a fleeting
relationship with software / electronics / robotics before should find the game
accessible enough that they are able to try several designs and score
``reasonable'' points. However, there should also be sufficient balance so that
those who are technically skilled are able to apply themselves.

\begin{enumerate}
\item The philosophy behind the game
\item The game selection process
\item The game revision and announcement process
\item Rule clarification process
\end{enumerate}
