\section{Game Design}

This section covers the philosophy behind the Student Robotics game, and the
procedures for designing and revising it.

\subsection{Game philosophy}

An ideal game would have:
\begin{itemize}
	\item Low barriers to entry
	\item Wide scope for technical achievement translating into points
	\item Enjoyable for those taking part as well as those spectating
\end{itemize}

for the purpose of giving the competitors something back for the effort they
put in as much as possible, while not setting the bar so low that there is no
incentive to develop technically challenging robot features.

To elaborate on these principles, for all Student Robotics games the following
invariants should hold:
\begin{enumerate}
	\item Contact between robots should not be part of the game. While the occasional collision is inevitable, we are not Robot Wars, and should discourage and penalise unsportsmanlike interference with other robots.
	\item Interaction: the competition is at it's best when robots are aware of each other and react to the others actions. While not encouraging contact, robots should not be able to play in isolation.
	\item Robots should move with a purpose. Part of the competition is developing robot software: the game should discourage random robot behaviour, requiring some level of intelligence.
	\item Reproducible in schools. Each team should be able to build a mock arena to test their robot without too much effort.
	\item Effectively implementable. The arena props and scoring process should be practical, requiring neither great expense or genius, although this should be considered in proportion to what additional expense or complexity will add to the game.
\end{enumerate}

Whilst encouraging the active exploration or development of the following
features by the competitiors, most commonly in the form of rewarding such
behaviours with additional points:
\begin{enumerate}
	\item Technical excellence: robots should be able to benefit from the development of non-trivial software, mechanical or electronic features.
	\item Variety: there should be multiple ways to solve the competition
		problem, avoiding all robots looking and behaving the same.
	\item Strategy: A robot that is able to better understand it's
		environment and plan ahead should score well.
	\item Robustness: the game and arena environment should not baby sit
		the competitors or otherwise make their lives easy. Dealing with
		uncertain and hazardous (to clumsy robots) environment is what
		makes the game interesting
\end{enumerate}

Finally, the game should be well grounded. The competitors are not all
programming geniuses or electronics wizards, and the aim of the competition is
to encourage them to study such things. Those who have only had a fleeting
relationship with software / electronics / robotics before should find the game
accessible enough that they are able to try several designs and score
``reasonable'' points. However, there should also be sufficient balance so that
those who are technically skilled are able to apply themselves.

\subsection{Game selection}

The process for selecting a game is inspired by the UK property development
planning application consultation process. The primary objective is to select
a game that fufils the philosophy above, with the additonal rider that we must
change the game each year to avoid vain repetition. The imagination to invent
such games comes from all Student Robotics volunteers, who use the procedure
below to propose games, and analyse other proposals, to help identify which
game best achieves the philosophy. To co-ordinate and provide certainty
regarding the best game, a game selection officer will be appointed by the
\roletitle{comp-prog-coord}. All participants will use the procedure outlined
below.

The resources (mailing lists, file hosting, discussion etc) used for game
selection will be determined by the game selection officer, but should be
accessible over the internet to all volunteers and \textbf{only} volunteers.
The game design should not be published before Kickstart.

\subsubsection{Game selection process}

The selection procedure has five main phases: informal discussions, proposals,
feedback, selection, and tweaking.

\begin{enumerate}
	\item \textbf{Informal discussions:} Before developing formal proposals
		for a game, volunteers should have the (optional) opportunity
		to sound out their ideas for feedback from other volunteers,
		to better understand their strengths and weaknesses. Greater
		exposure of an idea almost always leads to better analysis of
		how it can be improved. There is no formal structure for this
		phase (although it must be given start and end dates). There is
		also no ``ownership'' of a game idea (see below).
	\item \textbf{Proposals:} Any candidate game to be considered must be
		written as a formal proposal and submitted to the game
		selection officer for consideration. The deadline for the
		receipt of proposals should be announced well in advance to
		all volunteers, at a minimum giving at least one calendar weeks
		notice. Proposals must be submitted as a PDF and feature the
		following information:
		\begin{itemize}
			\item Name of game and author.
			\item Conceptual introduction: a description of what
				the technical challenges of the game are
				intended to be.
			\item Arena diagram: an illustration of what the arena
				would look like during the competition, with
				legend.
			\item List of props: what items need to be present
				in the arena and their features.
			\item Scoring scheme: an explanation of how the game
				is to be scored.
		\end{itemize}
		After the proposal deadline, the game selection officer will
		briefly examine proposals to ensure they meet these criteria
		and if necessary exclude any low quality submissions. If a
		proposal is obviously duplicate, this should be resolved too.
		Valid proposals should then be published to all volunteers.
	\item \textbf{Feedback:} All volunteers have the opportunity to read
		all proposals and comment on their various strengths and
		weaknesses. This feedback is public (to volunteers) and must
		not be anonymous. Moreover, volunteers may only comment on a
		proposal \textbf{once} to avoid un-necessary back-and-forth.
		Volunteers should fully cover any praise / criticism in their
		comments.

		The feedback period should be at least one calendar week, with
		comments submitted to the game selection officer to ensure the
		constraints above (alternately, a software solution that
		achives the same can be used).
	\item \textbf{Selection:} In receipt of all comments, the game
		selection officer should then read all the feedback for each
		proposal, and solicit responses to any criticism from the
		author. Based on the proposals, feedback and their responses, 
		the selection officer should select the proposal with the
		greatest potential for fufilling the game philosophy. The
		selected game should be announced within 1 week of the feedback
		period closing.
	\item \textbf{Revision:} After a proposal is selected, the author
		and selection officer should work together to revise the
		proposal to take account of feedback. They are jointly
		responsible for having a final rules document being
		written.
\end{enumerate}

\subsection{Game revision}

During the competition programme, a games maintainer will be appointed by the
programme coordinator. The maintainer should publish a mechanism for revisions
to the rules to be submitted to them, which after due consideration (including
feedback from the original game author) are to be published:
\begin{itemize}
	\item On the website
	\item On the forum
	\item To all mentors
\end{itemize}
