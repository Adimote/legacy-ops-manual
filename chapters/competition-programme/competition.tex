\section{The Competition}

The competition is a fun and exciting two-day event that is run to allow teams to play their robots against other robots.  The competition also allows teams to celebrate their achievements, meet other teams, and also discover what is possible.  Unlike kickstart, the competition is held in one location, thus maximising the spectacle of the event.

\subsection{Event Structure}

From a team's perspective, the competition consists of the following sections:

\begin{enumerate}
\item Arrival.

  Teams arrive at the venue, provide the paperwork that is necessary for their entry (i.e. tickets, or media consent forms), and are shown to the location of their pit, where they may work on their robot throughout the event.

\item Introduction.

  Teams are introduced to the event through a short introduction talk.  This talk gives teams an overview of how the event will run, and ensures that they are aware of relevant safety information.

\item League Matches.

  A series of matches is run, the results of which determine the position of teams within a league.  This forms the bulk of the matches that will be played at the competition, and ranks teams so they may be seeded into the knockout.  This seeding is arranged so that the top players should meet in the final match, to maximise the excitement throughout the event.

\item Photograph of teams.

  A gap is scheduled in the league matches so that a photograph of everyone in attendance can be taken.  This is scheduled somewhere near the middle of the league matches to ensure that all teams are still present (sadly, some teams choose to go home at the end of the league if they have not achieved a high position).

\item Knockout.

  A knockout tournament is held to determine the winner of the competition.

\item Awards ceremony.

  A guest speaker awards the winning teams with their prizes, and makes a short motivational speech.

\item Kit return.

  Teams return the loaned parts of their kit to SR.
\end{enumerate}

\subsection{Game Orchestration}

A large number of matches are scheduled to be played throughout each competition.  Many volunteers work together to ensure that these games run on schedule, are scored fairly, and are enjoyable for teams to participate in.

Since matches run on a tight schedule, left unchecked teams would turn up to many of their matches late, and therefore miss their opportunity to play in them.  The distraction of working on their robot is also a major factor here.  To help teams miss as few matches as possible, a ``shepherding'' scheme is run.  The scheme, run by the \role{head-shepherd}, aims to ensure that teams arrive at their matches on time.

Teams must turn up to their match in advance of it starting.  If they turn up later than a cut-off time before their match starts, then they must not be permitted to enter into the match.  Whilst this may seem harsh, it is generally the case that bringing teams in later into a match causes issues with matches having to be started late, which would reduce the enjoyment of the event for everyone else.  When teams arrive for their match, they should bring their robot into a ``staging area''.  Only one representative from each team should be allowed in this area at a time, and they should be tasked with placing their robot in the arena before the match begins.

The experience of watching a match should be enhanced by commentators.  The role of the commentators is to make the match more interesting to watch, maintaining the audience's interest and enthusiasm throughout each match.

The arenas in which the robots play the game need to be well lit throughout each match, so that spectators can see the robots and the robots can see the arena.  In between matches, lighting effects may be used to improve the contrast between when a match is running and when it is not.

Music should also be played throughout the event to bring an exciting atmosphere to the competition.  If songs are played so they begin with each match, then this further improves the contrast between matches.  Music should be used to increase the intensity of the competition so that the final knockout match is as enjoyable as possible.

The \role{head-match-scorer} is responsible for ensuring that the results of each of the games played throughout the event are collected.  These will ultimately be entered into the scoring system provided as part of the software managed by the \roletitle{comp-sw-ops-coord}.

No people should enter the arena whilst a match is in play.  However, there are some exceptional circumstances in which a volunteer, who is managed by the \roletitle{head-match-scorer}, may enter into the arena during a match to turn a robot off.  The circumstances in which a robot will need to be turned off during a match are as follows:
\begin{enumerate}
\item If a team ``throws in the towel'' during a match.  Note that teams must forfeit all points from a match if they do this.
\item If a robot is damaging another robot, the arena, or props within the arena.
\end{enumerate}
When entering the arena during a match to turn off a robot, the volunteer must be particularly careful not to disturb the actions of another robot.  Computer vision and other remote sensing technologies are employed by robots, so the volunteer should try to ensure they are not disrupting the data collected by those systems.

Sometimes uncertainty arises around the score that a team should be awarded for a match.  These uncertain situations must be resolved by one of the \roletitle{judges}.  In order to maintain consistency between these resolutions, judges must keep a record of them in a journal that is shared between all of the judges.

\subsection{Scheduling}

It is the responsibility of the \role{comp-prog-coord} to set the dates of the competition.  This is a tricky task.  There are a large number of attendees, and so there are a huge number of potential constraints as to when it could be.  Unfortunately it is not possible to satisfy all of the constraints of all  possible attendees, but through the application of the following constraints, it is possible to increase the liklihood that most will be able to attend:
\begin{enumerate}
\item The competition should be on a weekend in April.
\item The competition must not be scheduled on Easter weekend, and should not be scheduled on a bank holiday weekend.  These are likely to conflict with attendees holiday or family arrangements.
\end{enumerate}

It is important that the date of the competition is announced with significant notice.  This allows people to book hotels and travel while they are inexpensive, as well as letting them mark the date in their calendar so they do not schedule conflicting commitments.  The competition date should be announced at the kickstart that precedes it.

\subsection{Production}
\begin{enumerate}
\item Venue Requirements
    \begin{enumerate}
    \item Food and drink:
      \begin{enumerate}
      \item Can be purchased.
      \item Ideally not restricted to that sold at the venue -- it is free to participate in our event, and requiring people to buy something in particular at the event goes against this.
      \end{enumerate}
    \item Reasonable number of hotels nearby
    \item Reasonably straightforward to get to for all teams
    \item Sufficient space for each team to have an area they can work on their robot, which includes:
      \begin{enumerate}
      \item Chairs
      \item Desk
      \item Internet
      \item Power
      \end{enumerate}
    \item Venue must be heated reasonably during the event
    \item Adequate levels of parking must be available, and this must be able to accommodate a suitable number of minibuses.  This parking should be free for competitors, but does not have to be free to visitors. 
    \end{enumerate}

\item Production Details
  \begin{enumerate}
  \item ``Production'' covers things like:
    \begin{enumerate}
    \item Venue hire and management
    \item Equipment hire and purchase
    \item Logistics of getting equipment
    \item Dealing with third-party contractors
    \item Infrastructure installation
    \item Teardown
    \item Get-in
    \end{enumerate}

  \item Pits
    \begin{enumerate}
    \item Each team must be provided with a space that:
      \begin{enumerate}
      \item Is at least 3x3m
      \item Has 6 chairs available (Some teams will use more, some fewer)
      \item Has 2 power sockets
      \item Has a desk that is at least 1.5m$^2$
      \item Has either wireless or wired access to the internet
      \end{enumerate}

    \item Every pit desk must have a printed instruction sheet on regarding how the competition operates.  Includes:
      \begin{enumerate}
      \item Overall schedule
      \item Where to find match schedule
      \item How to get to a match
      \item How to get help
      \item First aid
      \item Emergency contact numbers (most likely of the Competition Event Coordinator)
      \item Any restrictions on using power tools, and messy paints/glues that apply during the event.
      \end{enumerate}
    \end{enumerate}

  \item Arena:
    \begin{enumerate}
    \item The arenas must be:
      \begin{enumerate}
      \item Positioned in venue to ensure easy viewing
      \item One arena will be designated as the one the finals will be viewed in.  The venue layout should be designed to allow more viewing area to this arena.

      \item Well lit with white light during matches.  Lighting effects must not be used during matches.
      \item Designed to permit teams to bring robots in and out
      \item Design to provide space to Match scorers
      \end{enumerate}
    \end{enumerate}

  \item Networking
    \begin{enumerate}
    \item Networking should preferably be installed by a third party contractor
    \item Every team should be given internet access throughout the event
    \item Areas requiring network:
      \begin{enumerate}
      \item Pits
      \item Arenas
      \item Entrance desks, and support desks.
      \end{enumerate}
    \end{enumerate}

  \item Radios
    \begin{enumerate}
    \item Radios should be provided to the following groups throughout the event:
      \begin{enumerate}
      \item Shepherds
      \end{enumerate}
    \end{enumerate}

  \item Power
    \begin{enumerate}
    \item Power distribution to all parts of competition must be installed by a suitably qualified and insured third-party contractor.
      \item All mains equipment that is used or hired by SR that is used at the competition must be PAT tested.
    \end{enumerate}

  \item Storage
    \begin{enumerate}
    \item The following items are owned by SR, and are used only at the competition:
      \begin{enumerate}
      \item Arena parts
      \end{enumerate}
      These parts must be transported to/from storage before/after the competition.
    \end{enumerate}


\item Health And Safety Requirements

  \begin{enumerate}
  \item All SR owned and hired equipment present at competition must be PAT tested.
  \item Risk assessment of event must be performed.
  \item First-aiders must be on-site.
  \end{enumerate}

  \end{enumerate}
  \end{enumerate}

\subsection{Volunteers}
  \begin{enumerate}
\item   \begin{enumerate}

\item Volunteer management
  \begin{enumerate}
  \item Volunteers should be provided with accommodation for the competition, as well as travel to/from the venue from/to that acommodation.
  \item Volunteers should be provided with transport to the competition from the town where they live. \textit{TODO: What constraints need to be put on this.  Cost?  Distance?  Number of volunteers?  Apply in advance?}

  \item Provide water throughout event
  \item Social event for volunteers should be organised on the Saturday evening of the event

  \end{enumerate}
  \end{enumerate}
  \end{enumerate}

\subsection{Tickets and Media Consent}
  \begin{enumerate}
  
\item Tickets and media consent:
  \begin{enumerate}
  \item Everyone who attends the competition, including volunteers, must have given consent for SR to use any photographs or videos that include them.  Those who do not give such consent must not be permitted to enter the competition venue.

  \item Since there are issues in verifying whether a competitor is a minor or not, \textbf{all} competitors must get their media consent form signed by their parent or guardian.  For the purposes of media consent, all competitors must be considered to be minors.

  \item In order to both speed up entry to the venue, and also to ensure that competitors do actually sort out their media consent forms, a competitor must receive a ticket from SR for the event prior to the event.  They must only receive this ticket if they have provided a signed, valid media consent form.  They will then use this ticket to gain entry to the venue.

  \item People who are clearly adults may be permitted entry to the venue if they sign a media consent form at the door.

  \item If minors arrive at the competition without a signed media consent form or ticket, and their parent or guardian is present, then their parent or guardian may sign a media consent on their behalf.

  \item Scans or photos of signed media consent forms may be transferred electronically to SR, provided that they are legible.

  \item The contents of the media consent form must be approved by the trustees.

  \item All media consent forms received for, or at, the competition must be digitised within one month after the competition.  These digitised forms must be sent to the trustees.  After the trustees have confirmed receipt, the physical copies must then be shredded to avoid distributing personal information.

  \item A system must be employed that allows re-entry to the venue to those who have already presented with a valid media consent arrangement.  This may be done with with a wristband system, or another reasonable approach.
  \end{enumerate}

  \end{enumerate}

\subsection{Awards}
\begin{enumerate}
\item Awards:
  \begin{enumerate}
  \item All awards must be described within the documentation provided to teams at kickstart.
  \item The following awards must exist:
    \begin{enumerate}
    \item First, second, and third prize.  The teams that receive these awards must be presented with trophies for achieving them.
    \end{enumerate}
  \item Other awards may also exist, and these should be decided upon by the \roletitle{game-design-coord}.
  \end{enumerate}
\end{enumerate}

\subsection{Kit return}
  \begin{enumerate}
  \item After the competition event, teams must return the kit that SR loaned to them.
  \item The kit may be returned in two ways:
    \begin{enumerate}
    \item At the end of the competition event itself.
    \item By courier before TODO days after the competition.  The courier service used by teams for this return must be sufficiently insured.  Teams must be provided with guidance on how to pack their kit for postage.
    \end{enumerate}
  \end{enumerate}


\subsection{Competition Software}

\begin{enumerate}
\item Throughout the event, everyone in attendance will want to have access to the following data:
  \begin{enumerate}
  \item Match schedule
  \item Scores:
    \begin{enumerate}
    \item League
    \item Knockout
    \end{enumerate}
  \end{enumerate}
  The competition software provides this to them.

\item The competition software must also:
  \begin{enumerate}
  \item Provide mechanism for information from match scorers to be entered, and amended.
  \item Allow for matches to be delayed as part of the schedule.
  \end{enumerate}

\item Match Scheduling:
  \begin{enumerate}
  \item The competition software should allow a team's matches to be scheduled to comply with the following requirements:
    \begin{enumerate}
    \item All matches in all arenas must begin simultaneously.
    \item A team's league matches should be well-distributed throughout the event.
    \item A team should face as many other teams as possible in matches throughout the league.
    \item Throughout the event, a team should experience a wide selection of initial conditions (e.g. the corner that they start in).
    \item The knockout:
      \begin{enumerate}
      \item Must begin with all teams
      \item Should begin in all arenas, but have quarter finals in only the main arena
      \end{enumerate}
    \item Time must be allocated between matches to allow for:
      \begin{enumerate}
      \item Scoring
      \item Arena reset
      \item Robot removal and installation
      \end{enumerate}
    \end{enumerate}
  \end{enumerate}

\end{enumerate}







This chapter should contain:
\begin{enumerate}

\item Team Management
  \begin{enumerate}
  \item Batteries and chargers off at beginning of event
  \end{enumerate}

\item Battery Charging (TODO: Include in Org chart)

\item Information that must be published in advance, and how far in advance each piece must be published:
  \begin{enumerate}
  \item Risk assessment for teams
  \item Match schedule
  \item Location
  \item Food
  \item Parking
  \end{enumerate}

\item Results publishing after event
\end{enumerate}
