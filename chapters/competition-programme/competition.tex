\section{The Competition}

The competition is a fun and exciting two-day event that is run to allow teams to play their robots against other robots.  The competition also allows teams to celebrate their achievements, meet other teams, and also discover what is possible.  Unlike kickstart, the competition is held in one location, thus maximising the spectacle of the event.

\subsection{Event Structure}

From a team's perspective, the competition consists of the following sections:

\begin{enumerate}
\item Arrival.

  Teams arrive at the venue, provide the paperwork that is necessary for their entry (i.e. tickets, or media consent forms), and are shown to the location of their pit, where they may work on their robot throughout the event.

\item Introduction.

  Teams are introduced to the event through a short introduction talk.  This talk gives teams an overview of how the event will run, and ensures that they are aware of relevant safety information.

\item Tinkering time.

  When teams arrive at the competition, they are invariably not ready to compete.  A short period of time is given to them so that they may get their robots into a state that is ready to compete.  During this time they may book time in the arenas for testing their robots.

\item League Matches.

  A series of matches is run, the results of which determine the position of teams within a league.  This forms the bulk of the matches that will be played at the competition, and ranks teams so they may be seeded into the knockout.  This seeding is arranged so that the top players should meet in the final match, to maximise the excitement throughout the event.

\item Photograph of teams.

  A gap is scheduled in the league matches so that a photograph of everyone in attendance can be taken.  This is scheduled somewhere near the middle of the league matches to ensure that all teams are still present (sadly, some teams choose to go home at the end of the league if they have not achieved a high position).

\item Knockout.

  A knockout tournament is held to determine the winner of the competition.

\item Awards ceremony.

  A guest speaker awards the winning teams with their prizes, and makes a short motivational speech.

\item Kit return.

  Teams return the loaned parts of their kit to SR.
\end{enumerate}

\subsection{Game Orchestration}

A large number of matches are scheduled to be played throughout each competition.  Many volunteers work together to ensure that these games run on schedule, are scored fairly, and are enjoyable for teams to participate in.

Since matches run on a tight schedule, left unchecked teams would turn up to many of their matches late, and therefore miss their opportunity to play in them.  The distraction of working on their robot is also a major factor here.  To help teams miss as few matches as possible, a ``shepherding'' scheme is run.  The scheme, run by the \role{head-shepherd}, aims to ensure that teams arrive at their matches on time.

Teams must turn up to their match in advance of it starting.  If they turn up later than a cut-off time before their match starts, then they must not be permitted to enter into the match.  Whilst this may seem harsh, it is generally the case that bringing teams in later into a match causes issues with matches having to be started late, which would reduce the enjoyment of the event for everyone else.  When teams arrive for their match, they should bring their robot into a ``staging area''.  Only one representative from each team should be allowed in this area at a time, and they should be tasked with placing their robot in the arena before the match begins.

The experience of watching a match should be enhanced by commentators.  The role of the commentators is to make the match more interesting to watch, maintaining the audience's interest and enthusiasm throughout each match.

The arenas in which the robots play the game need to be well lit throughout each match, so that spectators can see the robots and the robots can see the arena.  In between matches, lighting effects may be used to improve the contrast between when a match is running and when it is not.

Music should also be played throughout the event to bring an exciting atmosphere to the competition.  If songs are played so they begin with each match, then this further improves the contrast between matches.  Music should be used to increase the intensity of the competition so that the final knockout match is as enjoyable as possible.

The \role{head-match-scorer} is responsible for ensuring that the results of each of the games played throughout the event are collected.  These will ultimately be entered into the scoring system provided as part of the software managed by the \roletitle{comp-sw-coord}.

No people should enter the arena whilst a match is in play.  However, there are some exceptional circumstances in which a volunteer, who is managed by the \roletitle{head-match-scorer}, may enter into the arena during a match to turn a robot off.  The circumstances in which a robot will need to be turned off during a match are as follows:
\begin{enumerate}
\item If a team ``throws in the towel'' during a match.  Note that teams must forfeit all points from a match if they do this.
\item If a robot is damaging another robot, the arena, or props within the arena.
\end{enumerate}
When entering the arena during a match to turn off a robot, the volunteer must be particularly careful not to disturb the actions of another robot.  Computer vision and other remote sensing technologies are employed by robots, so the volunteer should try to ensure they are not disrupting the data collected by those systems.

Sometimes uncertainty arises around the score that a team should be awarded for a match.  These uncertain situations must be resolved by one of the \roletitle{judges}.  In order to maintain consistency between these resolutions, judges must keep a record of them in a journal that is shared between all of the judges.

\subsection{Scheduling}

It is the responsibility of the \role{comp-prog-coord} to set the dates of the competition.  This is a tricky task.  There are a large number of attendees, and so there are a huge number of potential constraints as to when it could be.  Unfortunately it is not possible to satisfy all of the constraints of all  possible attendees, but through the application of the following constraints, it is possible to increase the liklihood that most will be able to attend:
\begin{enumerate}
\item The competition should be on a weekend in April.
\item The competition must not be scheduled on Easter weekend, and should not be scheduled on a bank holiday weekend.  These are likely to conflict with attendees holiday or family arrangements.
\end{enumerate}

It is important that the date of the competition is announced with significant notice.  This allows people to book hotels and travel while they are inexpensive, as well as letting them mark the date in their calendar so they do not schedule conflicting commitments.  The competition date should be announced at the kickstart that precedes it.

\subsection{Production}

The ``production'' of the competition covers the procurement, organisation, set-up, and removal of all of the physical installations at the competition.  This includes arranging the contract with the venue, any licenses and insurance required for the event, as well as managing any third-party contractors who may be involved.

\subsubsection{Pits}

Each team should be provided with a ``pit'', which is a space at the competition that they can work on their robots in.  This space must comply with the following constraints:
\begin{enumerate}
\item It must be at least 3x3m.
\item It should feature 6 chairs for the team to use.  Some teams will require more, so there should be spare chairs available at the venue to allow this.
\item Provide 2 mains power sockets for the team to use.
\item Feature a desk that is at least 1.5m$^2$.
\item Provides the team that occupies it with either wireless or wired internet.
\end{enumerate}

In order to ensure that teams are aware of how the event runs, each pit desk should feature a printed instruction sheet that details how the competition operates.  This should be prepared by the \role{comp-team-coord}.  This sheet should include the following:
\begin{enumerate}
\item The overall schedule of the event.
\item Where to find the match schedule.
\item How to get to a match.
\item How to get help.
\item Emergency contact numbers.
\item First aid.
\item Any site rules -- this includes, for example, restrictions on using power tools or paints/glues.
\end{enumerate}

\subsubsection{Arenas}

The competition needs to feature at least one arena in which the robots should compete.  These arenas must be positioned within the venue to facilitate easy viewing.  Since the last few rounds of the finals are played in one arena, all of the competition attendees will want to be able to view that arena.  Adequate space should therefore be provided around that arenas.

Almost all of the robots in the competition rely on the arena being illuminated.  Therefore each arena must be well lit whilst a match is occurring.  Lighting effects may be used between matches, but in order to avoid interfering with robot vision systems, lighting effects must not be used during them.

The physical space around the arenas needs to be designed to permit teams to bring robots in and out of the arena.  It should also allow match scorers to view and access the arena throughout matches without being impeded by the audience.

\subsubsection{Networking}

Teams at the competition need a robust internet connection throughout the event.  This allows them to access the SR online services, as well as documentation from other websites, and also to post online about their participation in the event (e.g. on Twitter).

There are also several operations within the competition that require internet access, including the match scorers, and the entrance desks for ticket-checking purposes.  Volunteer providing support to teams will also need internet access so they may be as effective as possible.

It is preferable for the network within the venue to be installed by an external contractor.

\subsubsection{Radios}

Some of the volunteers at the competition will require a radio system to facilitate faster communication.  The shepherding team often benefits from the provision of such a system.

\subsubsection{Power}

Electric power is crucial to the running of just about every aspect of the event.  The power distribution system at the competition must be installed and managed by a suitably qualified and insured external contractor.  Furthermore, all mains equipment at the competition that is used or hired by SR must be PAT tested to ensure that it is safe.

\subsubsection{Storage}

There are several items owned by SR that are only used at a competition, including parts of the arenas.  These items should be kept in commercial storage between competitions.

\subsubsection{Health And Safety}

Assessments of the risks involved in all of the activities at the competition must be performed, and suitable control measures must be put in place to address the identified risks.

First-aiders must be on-site throughout the event to provide medical assistance.

\subsection{Venue Selection}

The \role{prod-manager} is responsible for finding and arranging a suitable venue for the competition.  There are several requirements that this venue must meet:
\begin{enumerate}
\item Teams must be able to purchase food and drink at the venue, or within a short distance of it.
\item Teams should be able to bring their own food and drink to the venue.  The competition is free to participate in for teams, and so SR should not force them to buy food from a specific vendor.
\item There should be a reasonable number of hotels nearby for both teams and volunteers to stay in.
\item The venue should be reasonably straightforward to get to for all teams.  This means that it should have reasonable travel links, and that it should be well-placed within the country.
\item The venue must have sufficient space for the team pits and arenas.
\item The venue must have suitable power, heating and lighting for the competition.
    \item Adequate levels of parking must be available, and this must be able to accommodate a suitable number of minibuses.  This parking should be free for competitors.
\end{enumerate}


\subsection{Volunteers}

SR's volunteers are somewhat geographically dispersed, and so it can be costly for them to attend the competition.  Therefore, SR should pay at least some of the travel expenses of those who volunteer at the competition.

Many volunteers also need to stay overnight near the competition venue.  SR should provide accommodation for these volunteers, as well as transport to get them between it and the venue.

During the event, volunteers will need to have regular refreshment, and this should be catered for.

A social event for volunteers should be run on the Saturday evening of the event weekend.

\subsection{Tickets and Media Consent}

SR uses media collected at the competition (e.g. photographs and videos) for publicity and other purposes.  Therefore everyone who attends the competition, including volunteers, must have given consent for SR to use any photographs or videos that include them.  Those who do not give such consent must not be permitted to enter the competition venue, as it is too challenging to manage otherwise.

Since there are issues in verifying whether a competitor is a minor or not, \textbf{all} competitors must get their media consent form signed by their parent or guardian.  For the purposes of media consent, all competitors must be considered to be minors.

In order to both speed up entry to the venue, and also to ensure that competitors do actually sort out their media consent forms, a competitor must receive a ticket from SR for the event prior to the event.  They must only receive this ticket if they have provided a signed, valid media consent form.   They will then use this ticket to gain entry to the venue.

Scans or photos of these signed media consent forms may be transferred electronically to SR, provided that they are legible.

People who are clearly adults may be permitted entry to the venue if they sign a media consent form at the door.

If minors arrive at the competition without a signed media consent form or ticket, and their parent or guardian is present, then their parent or guardian may sign a media consent on their behalf.

The contents of the media consent form must be approved by the trustees.

All media consent forms received for, or at, the competition must be digitised within one month after the competition.  These digitised forms must be sent to the trustees.  After the trustees have confirmed receipt, the physical copies must then be shredded to avoid distributing personal information.

A system must be employed that allows re-entry to the venue to those who have already presented with a valid media consent arrangement.  This may be done with with a wristband system, or another reasonable approach.

\subsection{Awards}

At the end of the competition, teams are given awards for their performance in the competition.  Some awards are also given for their performance throughout the year.  All of these awards must be described within the documentation that is provided to teams at kickstart.

The following awards must exist:
\begin{enumerate}
\item First, second, and third places in the competition.  The teams who receive these awards must be presented with trophies for achieving them.

\item Rookie award.  This should be presented to the team that ranked the highest in the competition's league and had not competed in SR before.

\item Robot and Team Image.  The team that presents the robot and themselves in what is judged to be the most outstanding way will receive this award.
\end{enumerate}
Other awards may exist, and these should be determined by the \role{game-design-coord}.

\subsection{Kit return}

Much of the kit that SR provides to teams is on loan to them.  This needs returning to SR after the competition.  It may be returned in one of two ways:

\begin{enumerate}
\item At the end of the competition event itself.
\item Returned by commercial courier by the 1\textsuperscript{st} June of the same year that the competition is being held in.  The courier service used by teams for this return must be sufficiently insured.  Teams must be provided with guidance on how to pack their kit for postage.
\end{enumerate}

\subsection{Competition Software}

At the competition, everyone in attendance needs to know various pieces of information about the things that are happening at the event.  These include:
\begin{enumerate}
\item The match schedule.  This is obviously particularly important to teams, as they need to know when to turn up to their matches.  It is also important to the team managed by the \role{comp-match-coord}.
\item The current scores of each of the teams, in both the league and the knockout.
\end{enumerate}
These pieces of information are provided to teams through the ``competition software'', which is managed by the \roletitle{comp-sw-coord}.  The match schedule must be available to teams before the competition begins.

The competition software must provide a mechanism through which the \role{head-match-scorer}, and their team, may enter and amend match scoring data from matches that have been played.  This scoring data should consist of arena state information, from which the software will derive the scores.

The competition software must provide a clock system that allows the audience of a match to see how much time is left in the current match.  This system must also indicate how much time is left before the next match starts, so that the match may start on time.

The competition software fulfils the important role of scheduling the matches that occur at the competition.  This match scheduler must comply with the following requirements:
\begin{enumerate}
\item All matches in all arenas must begin simultaneously.
\item A team's league matches should be well-distributed throughout the event.
\item A team should face as many other teams as possible in league matches.
\item Throughout the event, a team should experience a wide selection of initial conditions (e.g. the corner that they start in).
\item Time must be allocated between matches to allow for scoring, the resetting of arena props, and robot removal and installation.
\item The following requirements apply to the knockout:
  \begin{enumerate}
  \item The first round of the knockout must contain all teams, rather than a subset.
  \item The quarter finals, semi finals and finals must be scheduled to occur only in the main arena.  The other rounds of the knockout should use all available arenas.
  \end{enumerate}

\item The first few matches of the event must have more time between them to allow for any game orchestration issues to be handled.
\item The scheduler must provide a facility for delaying matches, so that if there are issues during the competition then the event can continue to run.  If the delay becomes significant, then the scheduler should fairly drop sets of matches to prevent the event from running late, whilst still adhering to all of the other constraints upon the match scheduler.
\end{enumerate}
