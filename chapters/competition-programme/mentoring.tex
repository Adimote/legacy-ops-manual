\section{Mentoring}

SR works with many 16-18 year-olds, the vast majority of whom have never participated in an engineering project before.  If left entirely to their own devices, many teams would give up before realising their full potential.  To avoid this from happening, SR provides teams with support through a mentoring programme.  Each team is assigned a mentor, who's role it is to help the team achieve their full potential within the competition.

Mentors must be 18 years old or over.  They must also possess skills that will help to further the teams' learning and understanding of engineering, science and team work. For example a mentor may be an engineering or science undergraduate, graduate or professional (note that this example is not an exhaustive list).

Since a mentor's main activity involves visiting a team on a regular basis, they will not generally meet other SR volunteers very often.  Therefore regular mentor socials are run by each \roletitle{local-team-coord}.  These socials allow mentors to discuss their mentoring activities with other mentors, as well as enjoy themselves!

\subsection{How to Mentor}
Every team that participates in SR is different, and so each team needs mentoring in a slightly different way.  What is crucial, in all situations, is that a mentor does not end up doing the team's work for them.  The mentor is a consultant, not a servant to the team.

When first starting to mentor a team, it is sensible to start off with a reasonably hands-off approach.  Start by asking the teams questions that are likely to make them think about the work that they have ahead of them, such as:
\begin{itemize}
\item How is the team going to organise itself?
\item What do they think their schedule is going to look like?
\item How much time do they actually have to spend on the project?
\item What sort of process are they going to use to develop their robot?
\end{itemize}

In order to successfully mentor a team, a mentor should meet up with that team on a regular basis.  This should ideally be in person, but does not have to be.  Mentoring by email, instant-messaging, etc. are all completely acceptable ways of assisting a team.  However, despite the hopes of many engineers and computer scientists, these technologies are still far less effective than meeting in person.  During these meetings, a mentor can help their team identify issues that they are having, such as:
\begin{itemize}
\item Technical issues (i.e. design and implementation problems)
\item Problems they are having with the SR kit or services.
\item Issues with coordinating their team.
\end{itemize}

Whilst a mentor can provide a great deal of support to a team, that team will benefit greatly if the mentor encourages them to participate in the online SR community.  Through this community, the team may interact with other teams and SR volunteers.  This provides them with an effective support system.

In order to ensure that SR provides the best service it can to teams, each mentor should report to their \roletitle{local-team-coord} when they have visited a team.  The \roletitle{local-team-coord} will be able to assist the mentor in their role, and ensure that the mentor is happy.

Sometimes, a mentor will not be able to make a meeting with their team.  Since teams sometimes fall into the belief that their mentor is their only source of support, it is important that the mentor makes their \roletitle{local-team-coord} aware that they are unable to attend.

From time-to-time, a mentor will encounter a problem that they do not know how to resolve.  When this situation arises, they should post a question on the forum, provided that it does not personal information.  When the forum is not appropriate, they should talk to their \roletitle{local-team-coord}, who will help them to work out how to resolve the issue.

As with all SR volunteers, mentors must comply with the DBS checking requirements described in section~\ref{sec:dbs-checking}.

\textit{The following paragraph is still being developed, as the maximum travel expense has not yet been set.}  SR will reimburse a mentor's travel expenses up to a maximum of {\pounds}N per week for travel to and from meetings with their team between the dates of kickstart and the competition.  These expense claims must comply with the requirements specified in chapter~\ref{chapter:money-matters}.

\subsection{Local Team Coordination}

Student Robotics has a growing number of teams, each of which requires support in the form of mentoring and processing of enquiries.  Providing this support is one of the most important aspects of SR's activities.  In order to make it manageable, the work is split into geographical localities.  The number of and shape of these localities is controlled by the \role{team-coord}, who appoints a \role{local-team-coord} to manage each of them.  Each \roletitle{local-team-coord} is responsible for supporting the set of teams from this locality, recruiting new teams in the area, and running socials for the mentors who support their teams.

If a volunteer is mentoring a team, then they are unfortunately reasonably isolated from the rest of SR.  To solve this problem, mentors should be given the opportunity to meet each other socially reasonably regularly.  In order to not significantly burden the mentor's schedule, these meetings should be appropriately infrequent (e.g. every month).  The socials shouldn't be complex in nature, making them straightforward to organise.  For example, a regular meeting in a pub is entirely appropriate.  Of course, other SR volunteers who are not mentors should also be invited to attend these socials.

It is the \roletitle{local-team-coord}'s responsibility to recruit teams into the competition from the area they have been allocated.  These teams will not be able to immediately start competing in the competition, as the programme is usually fully subscribed whilst it is running, so the teams should be prepared to register for the coming SR year.

Most of the teams are associated with schools, and so most teams are recruited from schools.  An approach that works well for recruiting schools is to send a letter to the headteacher at the school providing a brief description of what SR is, and the suggestion that the school may want to enter a team.  This letter should stress that entry into the competition is free, and that you are looking to get in touch with a teacher at the school who might be interested in running a team.  (\textit{TODO: A basic template for such a letter is provided in appendix X})  Usually, the school will not respond to this letter, as they are generally quite busy.  The next stage is to phone the school 2 to 4 weeks after sending the letter.  During this phonecall, one should try to get in touch with the head teacher to see if they received the letter, and subsequently try to get in touch with the teacher they suggest as a potential team leader.

Since the early days of recruiting schools in SR, schools have hopefully got considerably better at using email, so it may be worthwhile also trying to recruit schools by email.  Of course this may have less of an impact, but does raise the possibility of providing a link to a website into which a teacher can register their interest in having a phone-call from SR.

It will take some considerable time to recruit a school into the programme because teachers are generally very busy people.  Furthermore, teachers are usually on holiday throughout the summer, so recruitment should be performed months before the start of the summer holidays.
