\section{Mentoring}

\begin{itemize}
\item SR works with many 16-18 year-olds, the vast majority of whom have never participated in an engineering project before.  WE could leave them alone for the entirety of the SR year, but this would result in many giving up before realising their full potential.  It is a mentor's role to provide guidance and encouragement to participants.

\item Who can mentor:
  \begin{itemize}
  \item Mentors must be 18 or over.
  \item Mentors must meet at least one of the following requirements:
    \begin{itemize}
    \item Be enrolled in a science or engineering degree at university level.
    \item Have a science or engineering degree
    \item Be employed (inc self-employed, company director, etc.) as an engineer or scientist.
    \end{itemize}
  \end{itemize}
\end{itemize}

\subsection{How to Mentor}

\textit{Note that some of the following will be in the role description for mentors.}

\begin{itemize}
\item There are many ways to mentor a team, and the best strategy to employ will depend on the team itself.
\item Crucial not to do the team's work for them.
\item The mentor is a consultant to the team, not a servant.
\item Start with very hands-off approach, to see what sort of help the team may need.  As team things like:
  \begin{itemize}
  \item How they are organised
  \item What they think their schedule looks like.
  \item How much time they actually have to spend on the project.
  \end{itemize}
\item Mentors should regularly meet up with team:
  \begin{itemize}
  \item Ideally in person, at their school.  Ensure supervisor with DBS check (teacher) is present throughout.
  \item Mentors may communicate with their team via email, IM, etc. (\textit{TODO: A section covering regulations on school comms is needed.  Probably requires DBS check for such things.  Forums access may also require DBS check?})

  \item Mentors should encourage teams to participate in the online SR community, and not rely on the mentor as their only source of support.

  \item Mentors should help teams to identify issues they are having.  These may include, but are not limited to:
    \begin{itemize}
    \item Technical issues (e.g. design and implementation problems)
    \item Problems they are having with the SR kit or services.
    \item Issues with coordinating their team.
    \end{itemize}
  \end{itemize}

\item Mentors must report to their supervisor when they have visited a team, and inform them if they are unable to mentor the team for a significant period of time. (\textit{Move this to role description}).

\item Reporting problems:  If the team or mentor has a problem that they cannot resolve themselves, they should:
  \begin{itemize}
  \item Post on the forum if it is a problem that can be presented there (i.e. doesn't involve personal information)
  \item Talk to their supervisor (face-to-face, by email, or telephone, etc.)
  \end{itemize}

\item Conduct in schools: Some statement on the expected conduct of mentors in schools should be included here.

\item Travel Expenses:
  \begin{itemize}
  \item SR will reimburse a mentor's travel expenses up to a maximum of {\pounds}N per week for travel to and from meetings with their team between the dates of kickstart and the competition.  These expense claims must comply with the requirements specified in chapter X (money chapter).
  \end{itemize}

\end{itemize}

\subsection{Local Team Coordination}

\textit{This will be covered in the Local Team Coordinator's role section, but there should be some more detail here.}

\begin{itemize}
\item Local Mentor Socials:
  \begin{itemize}
  \item Mentors should be given the opportunity to meet each other socially reasonably regularly.  In order to not significantly burden the mentor's schedule, these meetings should be appropriately infrequent (e.g. every month).
  \item These meetings should be informal, and social in nature.
  \item Other local SR volunteers who are not mentors should also be invited to attend these socials.
  \end{itemize}

\item School Recruitment
  \begin{itemize}
  \item It is the \RoleLong{LTC}'s responsibility to recruit schools from the local area.
  \item Advice on recruiting schools:
    \begin{itemize}
    \item Send letters to schools as a start (\textit{Include a template letter in this document.}).  Follow up with a phone call after 2-4 weeks if no email response is received.
    \item Contact well in advance of expected start: will take many months to set up a team at a school.
    \end{itemize}
  \end{itemize}

\item Communications with schools
  \begin{itemize}
  \item It is the \RoleLong{LTC}'s responsibility to communicate with the team leader (\textit{``Team leader'' needs defining somewhere}), and to redirect their communications as appropriate.  \textit{Some kind of limits needs defining of the things that we will process from team leaders}.
  \end{itemize}

\end{itemize}
