\section{Teams}

The competing parties within the competition programme are referred to as ``teams''.  The teams should be comprised of 16-18 year olds.  Sometimes, younger people are interested in being a member of a team.  It is acceptable for these people to participate, as long as the majority of the team is comprised of 16-18 year-olds.  Occasionally, there may also be someone who is older than this range (e.g. because they repeated a year of school); it is also acceptable for them to participate.

Teams must have a minimum of 3 members, otherwise it may not participate in the competition programme.  The recommended number of members per team is 6, and teams may have many more than this.

Most of the teams that participate in SR are associated with an educational establishment (e.g. a college).  Teams do not need to have an association with a school, but such associations do aid teams in recruiting their members.

Every team must have a responsible adult associated with them, who is referred to as the ``team leader''.  This is usually a teacher from the school that the team is a part of.

The life-cycle of a team from SR's perspective is as follows:
\begin{enumerate}

\item SR announces that it is now taking applications for teams for the next SR year, and that applications must be filed by a given deadline date.  This opening of the application process should happen within a month of the last competition event.  Previous teams and potential teams should be notified that the registration process has begun, in order to encourage them to apply.

\item The team leader applies for a place in the upcoming SR competition.  They may have never participated in SR before, or they may have participated in previous years.

  In order to be allocated a place in the competition, a team must apply for that place, no matter how they were recruited or if they competed in previous years.

\item The application deadline is reached, and places in the competition are allocated to teams.  Those who are not allocated places are put on a wait-list, in case places become available during the year.

There are a limited number of places for teams in the competition programme, and these should be allocated on a first-come, first-served basis.  The trustees may provide further guidance on how teams are to be allocated -- for example, efforts may be made to increase diversity of teams, and so some teams may be prioritised over others.  Any such guidance that the trustees provide will be included in this section of this manual.

The following details must be collected from teams when they are awarded places in the competition:
\begin{itemize}
\item The team leader's name, email address, and telephone number.
\item The team leader's mobile phone number, so that they may be contacted in emergencies or whilst they are out of their school/office (e.g. while they are at the competition).
\item A postal address for the team.
\item Details of the organisation that the team is associated with.
\end{itemize}

\item Each of the teams that was awarded a place in the competition programme is assigned a TLA.  This TLA should be based on the name of the organisation that the team is associated with, or the geographical area that they are from.

  Teams that participate in multiple years of the competition should be assigned the same TLA every year.

\item The events of the SR year take place (kickstart, tech days, and the competition).
\end{enumerate}

If a team drops-out during the year, then efforts should be made to reallocate their kit to a new team who is not currently participating in the competition.  If there is no new team to replace them, then the kit should be provided to one of the largest currently-participating teams, so that they may split into two teams.
