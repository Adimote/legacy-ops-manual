\chapter{Volunteer Services}



\subsection{Volunteer Registration}

\begin{itemize}
\item All volunteers must be registered with SR as volunteers before they can volunteer for SR in any way.
\item Registration process is an online form that requires the following information:
  \begin{itemize}
  \item Name
  \item Telephone
  \item Email
  \item Address
  \item Emergency Contact Details
  \item A declaration that they understand they are not an employee of SR.
  \item A declaration that they will inform SR of any contact detail changes at the soonest point possible.
  \item They understand that by completing the form that they are not given permission to represent SR.  Such permissions are provided through role allocations, which are done later.
  \item Details of what they are looking to do -- including whether there is someone within the organisation who is actively recruiting them into a specific role.
  \end{itemize}
\end{itemize}

\subsection{Volunteer Recruitment}

\begin{itemize}
\item Coordinated by the \role{vol-rec-coord}
\item Volunteers may arrive at SR through a variety of routes.  If an existing volunteer meets someone who wishes to be a volunteer, they may:
  \begin{itemize}
  \item Refer them to the \role{vol-rec-coord}.
  \item Refer them to a specific individual within SR who is likely to be able to give them a role.
  \end{itemize}
Both routes will require them to register as described above.
  
\item CRB/DBS Requirements:
  \begin{itemize}
  \item Currently, volunteers are not required to have  DBS check to operate within SR.  The minors that SR works with are supervised at all times by a responsible adult who is DBS checked (e.g. a teacher).
  \item For situations where there is no DBS checked supervisor, and the volunteer is interacting with minors (i.e. people who are younger than 18), the volunteer must get a DBS check.
  \item DBS process instructions -- derive these from \url{https://www.gov.uk/disclosure-barring-service-check}
  \item \textit{Check through \url{https://www.gov.uk/guidance/charities-how-to-protect-vulnerable-groups-including-children}}
  \end{itemize}
\end{itemize}

\section{Aliases}

\begin{enumerate}
\item Some named roles will receive email and phone aliases from SR.  These make it easy for someone to get in touch with them, whilst preserving their privacy.  It also eases the transition when someone leaves a role.

\item Those with email and phone aliases should use those aliases when communicating with people on SR matters.

\end{enumerate}
