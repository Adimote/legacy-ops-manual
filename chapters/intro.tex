\chapter{Introduction}

This operations manual exists to provide Student Robotics\footnote{Student Robotics is a Charitable Incorporated Organisation (CIO) registered with the Charity Commission for England and Wales, with registration number 1163168.} volunteers with guidance so that they may work within Student Robotics (SR) to pursue its mission (see \autoref{sec:mission}).  It defines and describes every role that exists within the organisation, along with the powers that are associated with them.

It \roletel{VRC} is not expected that every volunteer should need to know the whole contents of this manual, but instead should be well versed in the sections that are relevant to their role.

It is likely that the contents of this manual will change over time.  The process through which this happens in detailed in section~\ref{sec:ops-man-update-procedure}.


\section{Mission and Vision}
\label{sec:mission}

\textit{The mission statement and vision statements will appear here.  These are going to be worked on by the trustees in the near future.}

\section{Definitions}

Throughout this document various terms will be used to refer to specific aspects of the organisation. Some of the most important of these are detailed below.

\begin{description}
  \item[The Competition] The two day event that forms the final part of the Competition Programme where Teams and their robots compete.
  \item[The Competition Programme] An annual autonomous robotics competition run by Student Robotics for 16-18 year olds.
  \item[The Competition Programme Year] The name used to refer to a particular Competition Programme. For example SR2016 refers to the Competition Programme running from Kickstart in November 2015 to The Competition in April 2016.
  \item[Event] A planned occasion for teams to attend. E.g. Kickstart, Tech Days, The Competition.
  \item[The Game] The challenge announced at Kickstart and played at The Competition.
  \item[Kickstart] A one day event that forms the start of the Competition Programme where teams are introduced to The Kit and The Game. Multiple Kickstart events are run simultaneously across the country.
  \item[The Kit] A collection of electronics hardware that is lent to teams for the period of the Competition Programme.
  \item[Team] The collective term for a group of team members and a team leader who compete in the Competition Programme.
  \item[Team Leader] An adult, usually, but not always, a teacher who is responsible for the team at Events.
  \item[Team Member] A person, generally in the age range of 16-18, who forms part of a team building a robot and competing in the Competition Programme.
  \item[Tech Day] A one day event that gives Teams a chance to work on and test their robots with access to advice from multiple mentors. It also serves as an opportunity to observe other Teams' progress.
\end{description}

\section{The Student Robotics Competition Programme}

The Student Robotics Competition Program forms the main part of Student Robotics and is how it currently meets its charitable objective set out in the constitution. It consists of an annual programme, aligned with the academic year, where teams of 16-18 year-olds partake in an reasonably open-ended engineering challenge to construct autonomous robots. Student Robotics provides support to teams in the form of loaned Kit and mentoring. The Competition Programme is covered in greater detail in \autoref{chapter:comp-prog}.
