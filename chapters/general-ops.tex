\chapter{General Operations}

SR's activities are divided into a hierarchical tree of \textit{roles}.  Each person who participates in SR's activities must be situated within at least one of these roles.  Their role describes their responsibilities, as well as the powers that they have to achieve them.

Some roles within the organisation have the responsibility of managing those in other roles.  The roles they are managing are said to be \textit{subordinate} to the management role.  Similarly, the management role is the \textit{superior} of the subordinate role.

This manual defines the role hierarchy, and the function and responsibilities of each role within it.  Each role that is described within this manual is given a name, and is referred to as a \textit{named role}.  Those in named roles may create new roles that are subordinate to themselves, as will be elaborated upon further in the next section.  Roles that are created in this manner are referred to as \textit{unnamed roles}.

\section{General Responsibilities}

Whilst there are responsibilities that are specific to individual roles, there are responsibilities that are common to every role within SR.  These general responsibilities are as follows:

\begin{enumerate}
  \item Providing periodic summaries of activities to one's superior via email, at a rate that is determined through conversation with one's superior.

  \item Responding to queries from one's superior.

  \item Working with others that one's superior has asked one to work with.

  \item Assisting one's superior in achieving one's superior's responsibilities as requested by one's superior.
\end{enumerate}

\subsection{Roles with Subordinates}
\label{sec:roles-with-subordinates}
Roles that have subordinates have the following responsibilities in addition to the general responsibilities listed above:

\begin{enumerate}
\item All responsibilities of the entire tree of subordinates below them in the organisational chart.

\item Selecting capable subordinates.

\item Keeping a list of the people who are currently in their subordinate roles.

\item Ensuring that all subordinates are suitably qualified for, and capable of, delivering their assigned responsibilities.  If a subordinate is incapable of delivering their assigned responsibilities then the superior should replace them with someone else.

\item Ensuring that all subordinates are suitably informed so that they can fulfil their responsibilities.

\item Delegating and managing responsibilities appropriately to immediate subordinates.

\item Ensuring that no subordinate is subject to stress or abuse due to their participation in this project.

\item Must produce documentation, as described in section~\ref{sec:documentation}.

\end{enumerate}

\subsection{Unnamed Role Creation}

Those in named roles may create hierarchies of new subordinate roles beneath them, and bring people into them as appropriate.  These new roles are to be generally referred to as \textit{unnamed roles}, since they are not named within this manual.  Note that all of the responsibilities described within this chapter are still applicable to those in unnamed roles.

Unnamed roles may only be responsible for a subset of the activities that they superior is responsible for, and the new role's responsibilities must not overlap with any other existing role.

The purpose and responsibilities of an unnamed role must be defined within the documentation published by the superior-to-be before anyone is given that role.

People in unnamed roles may define their own subordinate unnamed roles, but only if permitted by the documentation provided by their superior.

\subsection{Documentation}
\label{sec:documentation}

Those in roles with subordinates are required to produce documentation that details how their own role operates, including any systems or procedures that they set-up or control.  The purpose of this documentation is to both to assist one's subordinates, as well as to make it possible for someone else to take over one's own role in the event of one leaving it.

This documentation must be presented in a manner that is publicly accessible on the web, in a manner that is readable in either a web browser or a PDF viewer.  The documentation must be licensed under the \href{http://creativecommons.org/licenses/by-sa/4.0/}{Creative Commons Attribution-ShareAlike 4.0 license}, and must be stored in either a wiki system or a Git repository.

To make it possible to easily navigate through the organisation's documentation, this documentation must contain links to any documentation produced by one's immediate subordinates.  It should also contain links to one's superior's documentation for convenience.

It is the documentation producer's decision whether to permit others to edit their documentation, however edit-rights must not be given to those outside of the tree of the producer's subordinates.

The documentation produced by those with subordinate unnamed roles must cover the following aspects of those roles before anyone is brought into those roles:
\begin{enumerate}
\item The purpose and responsibilities of those roles.
\item How those roles are arranged with respect to each other in terms of hierarchy.
\item Any constraints, procedures or systems that are relevant to those roles.
\item Whether those roles may create further unnamed subordinate roles themselves.
\end{enumerate}

\section{Org Chart}
\label{sec:org-chart}

\begin{itemize}
\item The chart in Figure~\ref{fig:org-chart-top} shows the set of roles within the organisation, and their relation to their superiors and subordinates. This is not a complete organisational chart, but rather the set of top-level organisation wide roles. More detailed charts are shown in the relevent sections of this document.

\item Descriptions of the purpose of each role, along with their responsibilities, can be found in section~\ref{sec:genops-roles}.

\end{itemize}

\begin{figure}[h]
  \begin{center}
    \includegraphics[keepaspectratio,width=\textwidth]{org-chart/top.pdf}
  \end{center}
  \caption{\label{fig:org-chart-top}Top-level section of the organisational chart. The \roletitle{comp-prog-coord} role (circled in red) and the sub-tree below is covered in \autoref{sec:comp-prog-ops}.}
\end{figure}

\section{Roles}
\label{sec:genops-roles}
\begin{roledescriptions}{\subsection*}

\begin{roledesc}{trustees}
  The \roletitle{trustees} are responsible for all of SR's operations.  The role of the \roletitle{trustees} within the organisation is defined within the organisation's constitution.
\end{roledesc}

\begin{roledesc}{fund-coord}
  The \roletitle{fund-coord} is responsible for ensuring the organisation has sufficient funds to function.
  \begin{responsibilities}
  \item Sourcing funds.
  \item Managing all sponsorship arrangements, including those that involve the donation of goods not in the form of currency.
  \end{responsibilities}
\end{roledesc}

\begin{roledesc}{head-treasurer}
  The \roletitle{head-treasurer} is responsible for managing the organisation's finances.
  \begin{responsibilities}
    \item Maintaining the organisation's financial books.
    \item Complying with all financial reporting, tax, and audit requirements.
    \item Only spending money within the constraints of the organisation's budget.
  \end{responsibilities}
  \begin{services}
    \item Processing valid expense claims from volunteers, and reimbursing them as appropriate.
    \item Performing direct purchases on behalf of others in the organisation, when received from someone who is in a role that has clearance to spend that money.
  \end{services}
\end{roledesc}

\begin{roledesc}{internet-owner}
  The \roletitle{internet-owner} is responsible for the organisation's domain names.
\end{roledesc}

\begin{roledesc}{welfare-coord}
  The \roletitle{welfare-coord} is responsible for ensuring that all volunteers within the organisation are happy and healthy.
\end{roledesc}

\begin{roledesc}{vol-rec-coord}
The \roletitle{vol-rec-coord} is responsible for recruiting new volunteers and helping them to find a position in the organisation that suits them best.

The \roletitle{vol-rec-coord} should perform regular assessments of the volunteering requirements within the organisation, and adjust recruitment strategies to suit current demand.
\end{roledesc}

\begin{roledesc}{pr-coord}
  The \roletitle{pr-coord} is responsible for ensuring that the organisation is well-represented to the public.  This includes:
  \begin{responsibilities}
  \item Management of interaction with the press.
  \item The graphical style of SR.
  \end{responsibilities}
\end{roledesc}

\begin{roledesc}{pr-web-coord}
  The \roletitle{pr-web-coord} is responsible for the main SR website.
\end{roledesc}

\end{roledescriptions}
