\chapter{General Operations}


\begin{itemize}
\item Operations of SR are divided into a set of roles
\item Each role is assigned a set of responsibilities, some of which may include managing a set of subordinate roles
\item Every volunteer must be in a role to play a part in SR (there are no ``wildcard'' roles)

\item There is a set of general responsibilities that every person in a role has:
  \begin{itemize}
  \item Providing periodic summaries of activities to superior via email, at a rate that is determined through conversation with superior.
  \item Responding to queries from superior
  \item Working with others that their superior has required they work with
  \item Assisting their superior in achieving their superior's responsibilities as requested by their superior
  \end{itemize}
\item Roles with subordinates have these responsibilities too:
  \begin{itemize}
    \item All responsibilities of all individuals below them in the org chart
    \item Selecting capable subordinates if there are any in the org chart
    \item Keeping a list of people in subordinate roles
    \item Ensuring subordinates (if any) are:
      \begin{itemize}
      \item Suitably qualified for, and capable of, delivering their assigned responsibilities, and replaced if they are not
      \item Suitably informed
      \end{itemize}
    \item Delegating and managing responsibilities appropriately to immediate subordinates
    \item Ensuring that no subordinate is subject to stress or abuse due to their participation in this project
  \end{itemize}

\item People in roles named in this document (``named roles''):
  \begin{itemize}
  \item May create a hierarchy of new subordinate roles below them, and bring in new volunteers into them as appropriate.  These roles will be referred to as ``unnamed roles'', as they are not named in this document.  The following restrictions apply to these unnamed roles:
    \begin{itemize}
    \item The activities performed within these roles may only be a subset of the activities that their superior is assigned within this ops manual.
    \item People in unnamed roles may define their own subordinate unnamed roles, but only if permitted by the documentation provided by their superior (see below).
    \item A volunteer may only be brought into an unnamed role once that role is defined within the documentation produced by that role's superior.
    \end{itemize}

  \item Must produce documentation, which must:
    \begin{itemize}
    \item Detail the specifics of how their own role operates, including any systems and procedures they set-up or control.

    \item Cover the following regarding all unnamed roles they bring in:
      \begin{itemize}
      \item What the purpose and responsibilities of those unnamed roles are.
      \item Any procedures and/or systems that are relevant to those roles.

      \item The restrictions regarding whether that role may have unnamed subordinate roles of its own (e.g. none permitted, one level permitted, or more elaborate arrangements).
      \end{itemize}


  \item Be publicly readable on the web.

  \item Be able to be linked-to on the web.

  \item Link to any documentation produced by any subordinate roles.

  \item Be licensed under the Creative Commons BY-SA license.

  \item Be readable in at least one of the following:

    \begin{enumerate}
    \item A web browser
    \item A PDF viewer
    \end{enumerate}

  \item Be stored in one of the following types of system:
    \begin{itemize}
    \item A wiki
    \item A git repository
    \end{itemize}

    \end{itemize}

  \item It is the decision of the person in a named role whether to permit others to edit their documentation.  Edit-rights must be restricted to themselves and the tree of subordinates below them, or may be restricted further (e.g. only to the individual in the named role).

  \end{itemize}

\end{itemize}

\section{Org Chart}

\begin{itemize}
\item The chart in Figure~\ref{fig:org-chart-top} shows the set of roles within the organisation, and their relation to their superiors and subordinates. This is not a complete organisational chart, but rather the set of top-level organisation wide roles. More detailed charts are shown in the relevent sections of this document.

\item Descriptions of the purpose of each role, along with their responsibilities, can be found in section~\ref{sec:genops-roles}.

\end{itemize}

\begin{figure}[h]
  \begin{center}
    \includegraphics[keepaspectratio,width=\textwidth]{org-chart/top.pdf}
  \end{center}
  \caption{\label{fig:org-chart-top}Top-level section of the organisational chart. The \roletitle{comp-prog-coord} role (circled in red) and the sub-tree below is covered in \autoref{sec:comp-prog-ops}.}
\end{figure}

\section{Roles}
\label{sec:genops-roles}
\begin{roledescriptions}{\subsection*}

\begin{roledesc}{trustees}
\end{roledesc}

\begin{roledesc}{fund-coord}
The \roletitle{fund-coord} is responsible for ensuring the organisation has sufficient funds to function.
\end{roledesc}

\begin{roledesc}{head-treasurer}
  The \roletitle{head-treasurer} is responsible for managing the organisation's finances.
  \begin{responsibilities}
    \item Maintaining the organisation's financial books.
    \item Complying with all financial reporting, tax, and audit requirements.
    \item Only spending money within the constraints of the organisation's budget.
  \end{responsibilities}
  \begin{services}
    \item Processing valid expense claims from volunteers, and reimbursing them as appropriate.
    \item Performing direct purchases on behalf of others in the organisation, when received from someone who is in a role that has clearance to spend that money.
  \end{services}
\end{roledesc}

\begin{roledesc}{internet-owner}
  The \roletitle{internet-owner} is responsible for the organisation's domain names.
\end{roledesc}

\begin{roledesc}{welfare-coord}
  The \roletitle{welfare-coord} is responsible for ensuring that all volunteers within the organisation are happy and healthy.
\end{roledesc}

\begin{roledesc}{vol-rec-coord}
The \roletitle{vol-rec-coord} is responsible for recruiting new volunteers and helping them to find a position in the organisation that suits them best.
\end{roledesc}

\begin{roledesc}{pr-coord}
  The \roletitle{pr-coord} is responsible for ensuring that the organisation is well-represented to the public.  This includes:
  \begin{responsibilities}
  \item The management of the main SR website.
  \end{responsibilities}
\end{roledesc}

\end{roledescriptions}
